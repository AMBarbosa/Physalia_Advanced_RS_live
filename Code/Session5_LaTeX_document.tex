\documentclass[12pt, a4paper]{article}
\usepackage{graphicx} % Required for inserting images
\usepackage{hyperref}
\usepackage{natbib}
\usepackage{markdown}
% \usepackage{tcolorbox}


% to change line interspacing for submitting to journals:
% \renewcommand{\baselinestretch}{2}

\title{My LaTeX doc from Physalia RS course}
\author{A. Marcia Barbosa}
% \date{}
% if you don't provide or comment-out the \date{} argument, the document will show the computer date
% if you provide an empty \date{} argument, it will not appear


\begin{document}

\maketitle

\tableofcontents{}

\begin{abstract}
\noindent Just read the whole paper, it's not that long. I'd have chatGPT write me an abstract, but I can't be bothered to do even that.
\end{abstract}


\section{Introduction}
Intro... Certainly...

\begin{figure}
    \centering    \includegraphics[width=0.4\linewidth]{flare_las_2006347_lrg.jpg}
    \caption{Solar flare. Look at that!}
    \label{fig:flare}
\end{figure}

Another intro pgph... Certainly...

\noindent Non-indented intro pgph... Certainly...

We will see in section \ref{sec:methods} how to compute this.

\section{Methods}\label{sec:methods}

The code was something ike:

\begin{markdown}

```r
library(terra)
obj <- vect(c(1, 3)
```
\end{markdown}


The algorithm was based on the following steps:

\begin{itemize}
    \item Gathering data
    \item Cleaning data
    \item Analysing data
\end{itemize}

And now the same but with numbering:

\begin{enumerate}
    \item Gathering data
    \item Cleaning data
    \item Analysing data
\end{enumerate}


\subsection{Formulas}

In this paper, we used the formula of force by Newton, represented in Equation \ref{eq:newton}. And another more complex equation \ref{eq:complicadona} for show off. 

\begin{equation}
%F = \frac{G \times{m_1m_2}} {d^2}
F = G \times{ \frac{m_1} \times{m_2} {d^2}}
\label{eq:newton}
\end{equation}

\begin{equation}
    F = \sqrt[n]{G \times \frac{m_{13468} \times m_2}{r^{265756}}}
    \label{eq:complicadona}
\end{equation}


\subsection{Model calculation}

Models were computed with GLM, but they may not be right as I had a bump in my head (\ref{fig:galo}).

\begin{figure}
    \centering    \includegraphics[width=0.4\linewidth]{galoBarcelos.png}
    \caption{Nem todos os galos de Barcelos são de barro! Este é de testa.}
    \label{fig:galo}
\end{figure}


\subsection{Model evaluation}

Models were evaluated with modEvA...

\section{Results}

\subsection{Selected variables}

\subsection{Model convergence}

\subsection{Model evaluation}

Final results are shown in Figure \ref{fig:flare}

\section{Discussion}

Nothing much to discuss here, the conclusions are clear: you bump your head, you do a lot of nonsense. Nothing new, but still interesting nonetheless! See other papers for more sensible stuff \cite{Rocchini2018, Gallardo2024}.

\begin{thebibliography}{999}
\bibitem[Rocchini et al. (2018)]{Rocchini2018}
Rocchini D., Garzón-López C.X., Barbosa A.M., Delucchi L., Olandi J.E., Marcantonio M., Bastin L. \& Wegmann M. (2018) GIS-based Data Synthesis and Visualization. In: Recknagel F. \& Michener W. (eds.), Ecological Informatics: Data Management and Knowledge Discovery, 3rd Edition, pp. 273-286. Springer Verlag

\bibitem[Gallardo et al. (2024)]{Gallardo2024}
Gallardo B., Bacher S., Barbosa A.M., Gallien L., González-Moreno P., Martínez-Bolea V., Sorte C., Vimercati G. \& Vilà M. (2024) Risks posed by invasive species to the provision of ecosystem services in Europe. Nature Communications, 15: 2631
\end{thebibliography}


% add a horizontal line:
\hline

%\newpage

\smallskip
%\bigskip

\textbf{Box 1 - Steps to perform the analysis}
\begin{itemize}
\item blabla1
\item blabla2
\end{itemize}




\end{document}
